\documentclass[11pt,reqno]{beamer}
\usepackage[utf8x]{inputenc}
\usetheme{Dresden}
\usecolortheme{beaver}
\usepackage{amsmath}
\usepackage{amsfonts}
\usepackage{graphicx}
\usepackage{xcolor}
\usepackage{hyperref}


\setbeamertemplate{navigation symbols}{} 
\title{BGT software tutorial 0}
\subtitle{Installing the Jupyter stack}
\author{Peter Cudmore}

\institute{Systems Biology Lab, The University of Melbourne}

\newcommand{\D}[2]{\frac{\mathrm{d} #1}{\mathrm{d} #2}}
\newcommand{\e}{\mathrm{e}}
\newcommand{\I}{\mathrm{i}}
\renewcommand{\mod}[1]{\left|#1\right|}
\newcommand{\DD}[2]{\frac{\mathrm{d}^2 #1}{\mathrm{d} #2^2}}
\newcommand{\bigO}[1]{\text{O}\left(#1\right)}
\renewcommand{\P}[2]{\frac{\partial #1}{\partial #2}}
\renewcommand{\Re}{\operatorname{Re}}
\renewcommand{\Im}{\operatorname{Im}}
\newcommand{\EX}{\mathbb{E}}
\newcommand{\df}[1]{\mspace{2mu}  \mathrm{d}#1}
\newcommand{\reals}{\mathbb{R}}
\newcommand{\complex}{\mathbb{C}}
\newcommand{\conj}[1]{\overline{#1}}

\begin{document}
	\hypersetup{urlcolor=blue, linkcolor=blue}
	\begin{frame}
	\titlepage
	\addtocounter{framenumber}{-1} 
\end{frame}
\begin{frame}
\tableofcontents[hideallsubsections]
\end{frame}
\section{Goals}
\begin{frame}
\begin{itemize}
	\item Install Python 3.6 or 3.7
	\item Install Jupyter notebook
	\item Install Julia 0.6
	\item Make sure Julia talks to Python and Jupyter
\end{itemize}
\end{frame}
\begin{frame}{Download and Install Python}
\emph{We will not be using venv or virtual env} 
\begin{itemize}
	\item All platforms: Install anaconda
	\item On windows: \url{https://www.python.org/downloads/windows/} install 3.6.5 or 3.7
	\item On Mac: Install homebrew \url{https://brew.sh/} then python \texttt{>brew install python} 
	\item On Linux; use your package manager to install 3.6 or 3.7 ALONG SIDE distribution version
\end{itemize}
\end{frame}
\begin{frame}{Make sure Python is in your PATH}
Here \texttt{(PYTHON\_DIR)} is the directory where you install python. (For example 
\texttt{C:\slash Program FIles\slash Python3.7 } or \texttt{/opt/python}
\vfill
For windows:\\
Following \href{https://helpdeskgeek.com/windows-10/add-windows-path-environment-variable/}{this guide}, add your python folder to the PATH variables.
\vfill
Otherwise add the following line \texttt{\textasciitilde\slash.bashrc} (linux) or \texttt{\textasciitilde\slash.bash\_profile} (osx):\\
\vspace{10pt}
\texttt{set PATH=(PYTHON\_DIR)\slash bin:\$PATH}\\
\vfill
\end{frame}
\begin{frame}{Install Jupyter}

Windows: Using the SAME python interpreter run \texttt{>python -m pip install jupyter}
\vfill
On osx: Using homebrew\\
 \texttt{>brew install jupyter}\\
\vfill
(On Ubuntu) Use apt to install\\
\texttt{>sudo apt install jupyter-notebook}\\
\vfill

test this by running \texttt{>jupyter-notebook}
\end{frame}
\begin{frame}{Install Julia}
Install Julia 0.6.4 for your operating system from:\\
\url{https://julialang.org/downloads/oldreleases.html}\\
\vfill

Make sure to add julia to your PATH variable as you did with python.

\vfill 
For windows 7 users; make sure to install the additional fixes listed.
\end{frame}
\begin{frame}{Install Julia  Binding}
Install Ijulia with
\texttt{>Pkg.add("IJulia")}\\
then exit julia with \texttt{>exit()}
\vfill
Test by opening
\texttt{jupyter-notebook}\\
and creating a new julia notebook.
\end{frame}




\end{document}
